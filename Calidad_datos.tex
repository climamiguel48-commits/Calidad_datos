% Options for packages loaded elsewhere
% Options for packages loaded elsewhere
\PassOptionsToPackage{unicode}{hyperref}
\PassOptionsToPackage{hyphens}{url}
\PassOptionsToPackage{dvipsnames,svgnames,x11names}{xcolor}
%
\documentclass[
  spanish,
  11pt,
  a4paper,
]{article}
\usepackage{xcolor}
\usepackage[margin=2.5cm]{geometry}
\usepackage{amsmath,amssymb}
\setcounter{secnumdepth}{5}
\usepackage{iftex}
\ifPDFTeX
  \usepackage[T1]{fontenc}
  \usepackage[utf8]{inputenc}
  \usepackage{textcomp} % provide euro and other symbols
\else % if luatex or xetex
  \usepackage{unicode-math} % this also loads fontspec
  \defaultfontfeatures{Scale=MatchLowercase}
  \defaultfontfeatures[\rmfamily]{Ligatures=TeX,Scale=1}
\fi
\usepackage{lmodern}
\ifPDFTeX\else
  % xetex/luatex font selection
\fi
% Use upquote if available, for straight quotes in verbatim environments
\IfFileExists{upquote.sty}{\usepackage{upquote}}{}
\IfFileExists{microtype.sty}{% use microtype if available
  \usepackage[]{microtype}
  \UseMicrotypeSet[protrusion]{basicmath} % disable protrusion for tt fonts
}{}
\makeatletter
\@ifundefined{KOMAClassName}{% if non-KOMA class
  \IfFileExists{parskip.sty}{%
    \usepackage{parskip}
  }{% else
    \setlength{\parindent}{0pt}
    \setlength{\parskip}{6pt plus 2pt minus 1pt}}
}{% if KOMA class
  \KOMAoptions{parskip=half}}
\makeatother
% Make \paragraph and \subparagraph free-standing
\makeatletter
\ifx\paragraph\undefined\else
  \let\oldparagraph\paragraph
  \renewcommand{\paragraph}{
    \@ifstar
      \xxxParagraphStar
      \xxxParagraphNoStar
  }
  \newcommand{\xxxParagraphStar}[1]{\oldparagraph*{#1}\mbox{}}
  \newcommand{\xxxParagraphNoStar}[1]{\oldparagraph{#1}\mbox{}}
\fi
\ifx\subparagraph\undefined\else
  \let\oldsubparagraph\subparagraph
  \renewcommand{\subparagraph}{
    \@ifstar
      \xxxSubParagraphStar
      \xxxSubParagraphNoStar
  }
  \newcommand{\xxxSubParagraphStar}[1]{\oldsubparagraph*{#1}\mbox{}}
  \newcommand{\xxxSubParagraphNoStar}[1]{\oldsubparagraph{#1}\mbox{}}
\fi
\makeatother

\usepackage{color}
\usepackage{fancyvrb}
\newcommand{\VerbBar}{|}
\newcommand{\VERB}{\Verb[commandchars=\\\{\}]}
\DefineVerbatimEnvironment{Highlighting}{Verbatim}{commandchars=\\\{\}}
% Add ',fontsize=\small' for more characters per line
\usepackage{framed}
\definecolor{shadecolor}{RGB}{241,243,245}
\newenvironment{Shaded}{\begin{snugshade}}{\end{snugshade}}
\newcommand{\AlertTok}[1]{\textcolor[rgb]{0.68,0.00,0.00}{#1}}
\newcommand{\AnnotationTok}[1]{\textcolor[rgb]{0.37,0.37,0.37}{#1}}
\newcommand{\AttributeTok}[1]{\textcolor[rgb]{0.40,0.45,0.13}{#1}}
\newcommand{\BaseNTok}[1]{\textcolor[rgb]{0.68,0.00,0.00}{#1}}
\newcommand{\BuiltInTok}[1]{\textcolor[rgb]{0.00,0.23,0.31}{#1}}
\newcommand{\CharTok}[1]{\textcolor[rgb]{0.13,0.47,0.30}{#1}}
\newcommand{\CommentTok}[1]{\textcolor[rgb]{0.37,0.37,0.37}{#1}}
\newcommand{\CommentVarTok}[1]{\textcolor[rgb]{0.37,0.37,0.37}{\textit{#1}}}
\newcommand{\ConstantTok}[1]{\textcolor[rgb]{0.56,0.35,0.01}{#1}}
\newcommand{\ControlFlowTok}[1]{\textcolor[rgb]{0.00,0.23,0.31}{\textbf{#1}}}
\newcommand{\DataTypeTok}[1]{\textcolor[rgb]{0.68,0.00,0.00}{#1}}
\newcommand{\DecValTok}[1]{\textcolor[rgb]{0.68,0.00,0.00}{#1}}
\newcommand{\DocumentationTok}[1]{\textcolor[rgb]{0.37,0.37,0.37}{\textit{#1}}}
\newcommand{\ErrorTok}[1]{\textcolor[rgb]{0.68,0.00,0.00}{#1}}
\newcommand{\ExtensionTok}[1]{\textcolor[rgb]{0.00,0.23,0.31}{#1}}
\newcommand{\FloatTok}[1]{\textcolor[rgb]{0.68,0.00,0.00}{#1}}
\newcommand{\FunctionTok}[1]{\textcolor[rgb]{0.28,0.35,0.67}{#1}}
\newcommand{\ImportTok}[1]{\textcolor[rgb]{0.00,0.46,0.62}{#1}}
\newcommand{\InformationTok}[1]{\textcolor[rgb]{0.37,0.37,0.37}{#1}}
\newcommand{\KeywordTok}[1]{\textcolor[rgb]{0.00,0.23,0.31}{\textbf{#1}}}
\newcommand{\NormalTok}[1]{\textcolor[rgb]{0.00,0.23,0.31}{#1}}
\newcommand{\OperatorTok}[1]{\textcolor[rgb]{0.37,0.37,0.37}{#1}}
\newcommand{\OtherTok}[1]{\textcolor[rgb]{0.00,0.23,0.31}{#1}}
\newcommand{\PreprocessorTok}[1]{\textcolor[rgb]{0.68,0.00,0.00}{#1}}
\newcommand{\RegionMarkerTok}[1]{\textcolor[rgb]{0.00,0.23,0.31}{#1}}
\newcommand{\SpecialCharTok}[1]{\textcolor[rgb]{0.37,0.37,0.37}{#1}}
\newcommand{\SpecialStringTok}[1]{\textcolor[rgb]{0.13,0.47,0.30}{#1}}
\newcommand{\StringTok}[1]{\textcolor[rgb]{0.13,0.47,0.30}{#1}}
\newcommand{\VariableTok}[1]{\textcolor[rgb]{0.07,0.07,0.07}{#1}}
\newcommand{\VerbatimStringTok}[1]{\textcolor[rgb]{0.13,0.47,0.30}{#1}}
\newcommand{\WarningTok}[1]{\textcolor[rgb]{0.37,0.37,0.37}{\textit{#1}}}

\usepackage{longtable,booktabs,array}
\usepackage{calc} % for calculating minipage widths
% Correct order of tables after \paragraph or \subparagraph
\usepackage{etoolbox}
\makeatletter
\patchcmd\longtable{\par}{\if@noskipsec\mbox{}\fi\par}{}{}
\makeatother
% Allow footnotes in longtable head/foot
\IfFileExists{footnotehyper.sty}{\usepackage{footnotehyper}}{\usepackage{footnote}}
\makesavenoteenv{longtable}
\usepackage{graphicx}
\makeatletter
\newsavebox\pandoc@box
\newcommand*\pandocbounded[1]{% scales image to fit in text height/width
  \sbox\pandoc@box{#1}%
  \Gscale@div\@tempa{\textheight}{\dimexpr\ht\pandoc@box+\dp\pandoc@box\relax}%
  \Gscale@div\@tempb{\linewidth}{\wd\pandoc@box}%
  \ifdim\@tempb\p@<\@tempa\p@\let\@tempa\@tempb\fi% select the smaller of both
  \ifdim\@tempa\p@<\p@\scalebox{\@tempa}{\usebox\pandoc@box}%
  \else\usebox{\pandoc@box}%
  \fi%
}
% Set default figure placement to htbp
\def\fps@figure{htbp}
\makeatother



\ifLuaTeX
\usepackage[bidi=basic]{babel}
\else
\usepackage[bidi=default]{babel}
\fi
% get rid of language-specific shorthands (see #6817):
\let\LanguageShortHands\languageshorthands
\def\languageshorthands#1{}


\setlength{\emergencystretch}{3em} % prevent overfull lines

\providecommand{\tightlist}{%
  \setlength{\itemsep}{0pt}\setlength{\parskip}{0pt}}



 


\makeatletter
\@ifpackageloaded{caption}{}{\usepackage{caption}}
\AtBeginDocument{%
\ifdefined\contentsname
  \renewcommand*\contentsname{Tabla de contenidos}
\else
  \newcommand\contentsname{Tabla de contenidos}
\fi
\ifdefined\listfigurename
  \renewcommand*\listfigurename{Listado de Figuras}
\else
  \newcommand\listfigurename{Listado de Figuras}
\fi
\ifdefined\listtablename
  \renewcommand*\listtablename{Listado de Tablas}
\else
  \newcommand\listtablename{Listado de Tablas}
\fi
\ifdefined\figurename
  \renewcommand*\figurename{Figura}
\else
  \newcommand\figurename{Figura}
\fi
\ifdefined\tablename
  \renewcommand*\tablename{Tabla}
\else
  \newcommand\tablename{Tabla}
\fi
}
\@ifpackageloaded{float}{}{\usepackage{float}}
\floatstyle{ruled}
\@ifundefined{c@chapter}{\newfloat{codelisting}{h}{lop}}{\newfloat{codelisting}{h}{lop}[chapter]}
\floatname{codelisting}{Listado}
\newcommand*\listoflistings{\listof{codelisting}{Listado de Listados}}
\makeatother
\makeatletter
\makeatother
\makeatletter
\@ifpackageloaded{caption}{}{\usepackage{caption}}
\@ifpackageloaded{subcaption}{}{\usepackage{subcaption}}
\makeatother
\usepackage{bookmark}
\IfFileExists{xurl.sty}{\usepackage{xurl}}{} % add URL line breaks if available
\urlstyle{same}
\hypersetup{
  pdftitle={Informe de control de calidad de datos climáticos - Parcela 3},
  pdfauthor={Ing. Miguel Silva},
  pdflang={es},
  colorlinks=true,
  linkcolor={blue},
  filecolor={Maroon},
  citecolor={Blue},
  urlcolor={Blue},
  pdfcreator={LaTeX via pandoc}}


\title{\textbf{Informe de control de calidad de datos climáticos -
Parcela 3}}
\author{Ing. Miguel Silva}
\date{2025-12-06}
\begin{document}
\maketitle

\renewcommand*\contentsname{Tabla de contenidos}
{
\hypersetup{linkcolor=}
\setcounter{tocdepth}{3}
\tableofcontents
}

\section{Introducción}\label{introducciuxf3n}

En este informe se presenta el control de calidad de la base de datos
climáticos registrada en la Parcela 3, a partir del archivo
\texttt{Datos\_Parcela\_3.xlsx}.

El objetivo principal es identificar y cuantificar:

\begin{itemize}
\item
  Datos faltantes por variable.\\
\item
  Valores fuera de rangos físicos razonables.\\
\item
  Registros atípicos que puedan comprometer análisis posteriores
  (balance de energía, estimación de ET, etc.).
\item
  Visualizar el comportamiento temporal de las variables.
\end{itemize}

\section{Descripción general de la base de
datos}\label{descripciuxf3n-general-de-la-base-de-datos}

La base de datos esta conformada por los registros horarios de las
siguientes variables:

\textbf{Variables meteorológicas}:

\begin{itemize}
\item
  Temperatura (°C)
\item
  Humedad relativa (\%)
\item
  Velocidad del viento (m/s)
\item
  Dirección del viento (°)
\item
  Presión atmosférica (mbar)
\item
  Radiación global (Wm-2)
\item
  Precipitación (mm)
\end{itemize}

\textbf{Variables de flujos de energía}:

\begin{itemize}
\item
  Temperatura superficial (°C)
\item
  Flujo de energía en el suelo (Wm-2)
\item
  Radiación Neta (Wm-2)
\item
  Temperatura termocupla 1 (°C)
\item
  Temperatura termocupla 2 (°C)
\item
  Humedad del suelo a los 5 cm (m3/m3)
\item
  Humedad del suelo a los 40 cm (m3/m3)
\end{itemize}

Para su procesamiento preliminar se aplica lo siguiente:

\begin{verbatim}
# A tibble: 1 x 3
  fecha_inicio        fecha_fin           n_registros
  <dttm>              <dttm>                    <int>
1 2025-07-21 14:00:00 2025-11-28 15:00:00        2438
\end{verbatim}

\begin{verbatim}
# A tibble: 1 x 9
  n_Temperatura_aire_C n_Humedad_relativa_porc n_Precipitacion_mm
                 <int>                   <int>              <int>
1                 2434                    2434               2434
# i 6 more variables: n_Velocidad_viento_ms <int>, n_Radiacion_neta_Wm2 <int>,
#   n_Radiacion_global_Wm2 <int>, n_Temperatura_superficie_C <int>,
#   n_Humedad_suelo_5cm_m3m3 <int>, n_Humedad_suelo_40cm_m3m3 <int>
\end{verbatim}

\section{Análisis exploratorio}\label{anuxe1lisis-exploratorio}

Este análisis primero identifica el período temporal cubierto, el número
de registros y variables, y determina qué columnas contienen información
numérica. Luego calcula estadísticas descriptivas básicas; media,
mediana, desviación estándar, mínimos, máximos y cantidad de datos
faltantes para todas las variables, permitiendo detectar rangos típicos,
dispersión y posibles anomalías. Finalmente, cuantifica los valores
ausentes por variable, lo cual ayuda a evaluar la completitud del
conjunto de datos antes de aplicar controles de calidad más estrictos.

\section{Control de calidad}\label{control-de-calidad}

Las reglas de control de calidad aplicadas en este informe son:

\textbf{Continuidad temporal}

Verificación de que la serie está ordenada y sin duplicados en
fecha\_hora.

Rangos físicos razonables (pueden ajustarse a tus criterios locales):

Temperatura\_aire\_C: -10 a 45 °C

Humedad\_relativa\_porc: 0 a 100 \%

Precipitacion\_mm: ≥ 0 mm

Velocidad\_viento\_ms: ≥ 0 m/s

Radiacion\_neta\_Wm2: -200 a 1000 W/m²

\textbf{Tratamiento de datos}

Los valores fuera de rango se marcan mediante flags y, en la base
validada, se reemplazan por NA para evitar su uso directo en análisis.

\section{Resultados}\label{resultados}

\textbf{Resultados del análisis exploratorio}

\begin{Shaded}
\begin{Highlighting}[]
\NormalTok{resultados\_ae }\OtherTok{\textless{}{-}} \FunctionTok{analisis\_exploratorio}\NormalTok{(datos\_brutos)}

\CommentTok{\#Info básica en tabla}

\NormalTok{info\_basica }\OtherTok{\textless{}{-}} \FunctionTok{as.data.frame}\NormalTok{(resultados\_ae}\SpecialCharTok{$}\NormalTok{info\_basica)}
\NormalTok{info\_basica}
\end{Highlighting}
\end{Shaded}

\begin{verbatim}
  n_filas n_columnas fecha_inicio  fecha_fin duracion_dias
1    2438         32   1753106400 1764342000      130.0417
\end{verbatim}

\begin{Shaded}
\begin{Highlighting}[]
\CommentTok{\#| label: estadisticas{-}principales}
\CommentTok{\#| echo: true}

\NormalTok{estadisticas\_df }\OtherTok{\textless{}{-}}\NormalTok{ purrr}\SpecialCharTok{::}\FunctionTok{map\_dfr}\NormalTok{(}
\NormalTok{resultados\_ae}\SpecialCharTok{$}\NormalTok{estadisticas,}
\SpecialCharTok{\textasciitilde{}} \FunctionTok{as.data.frame}\NormalTok{(.x),}
\AttributeTok{.id =} \StringTok{"variable"}
\NormalTok{)}

\NormalTok{estadisticas\_df}
\end{Highlighting}
\end{Shaded}

\begin{verbatim}
                    variable         media   mediana   desviacion       min
1       Radiacion_global_Wm2  1.891088e+02   10.8350 2.617007e+02     0.000
2           Precipitacion_mm  2.940427e-02    0.0000 2.730409e-01     0.000
3               Rayos_conteo  2.875924e-03    0.0000 8.354031e-02     0.000
4        Velocidad_viento_ms  5.160140e-01    0.4580 3.449411e-01     0.048
5    Direccion_viento_grados  1.761297e+02  163.9000 7.951649e+01     0.900
6                  Rachas_ms  1.511889e+00    1.3470 1.043812e+00     0.123
7         Temperatura_aire_C  1.559380e+01   14.2250 6.835956e+00     2.317
8   Presion_atmosferica_mbar  1.016620e+03 1016.3310 3.036572e+00  1009.648
9      Humedad_relativa_porc  6.761639e+01   70.6500 2.316700e+01    15.700
10     Radiacion_global_MJm2  6.816402e-01    0.0366 9.424293e-01     0.000
11  Temperatura_superficie_C  1.596208e+01   13.7500 8.175708e+00     1.126
12   Flujo_energia_suelo_Wm2 -9.835949e-02   -1.0150 4.760450e+00   -10.820
13        Radiacion_neta_Wm2  5.264392e+02  317.9000 7.687920e+02 -1642.000
14 Temperatura_termocupla1_C  1.658387e+01   14.1500 8.853681e+00     0.719
15 Temperatura_termocupla2_C  1.658374e+01   14.1500 8.853524e+00     0.721
16    Humedad_suelo_5cm_m3m3  3.570538e-01    0.3650 3.624735e-02     0.214
17   Humedad_suelo_40cm_m3m3  4.897650e-01    0.4910 5.753877e-03     0.476
         max nas
1   873.0000   0
2     6.0350   4
3     3.0000   4
4     2.1030   4
5   359.7000   4
6     6.3280   4
7    32.9200   4
8  1027.0980   4
9   100.0000   4
10    3.1428   4
11   36.3800   4
12   12.4600   4
13 3004.0000   4
14   38.4300   4
15   38.4300   4
16    0.4500   4
17    0.5010   4
\end{verbatim}

\begin{Shaded}
\begin{Highlighting}[]
\CommentTok{\#| label: valores{-}faltantes}
\CommentTok{\#| echo: true}

\NormalTok{valores\_faltantes\_df }\OtherTok{\textless{}{-}} \FunctionTok{data.frame}\NormalTok{(}
\AttributeTok{variable =} \FunctionTok{names}\NormalTok{(resultados\_ae}\SpecialCharTok{$}\NormalTok{valores\_faltantes),}
\AttributeTok{n\_na =} \FunctionTok{as.numeric}\NormalTok{(resultados\_ae}\SpecialCharTok{$}\NormalTok{valores\_faltantes)}
\NormalTok{)}

\NormalTok{valores\_faltantes\_df}
\end{Highlighting}
\end{Shaded}

\begin{verbatim}
                    variable n_na
1                 fecha_hora  105
2           Precipitacion_mm    4
3               Rayos_conteo    4
4                Dist_km_Avg    4
5        Velocidad_viento_ms    4
6    Direccion_viento_grados    4
7                  Rachas_ms    4
8         Temperatura_aire_C    4
9                VP_mbar_Avg    4
10  Presion_atmosferica_mbar    4
11                      ETos    4
12                       Rso    4
13     Humedad_relativa_porc    4
14                     RHT_C    4
15            TiltNS_deg_Avg    4
16            TiltWE_deg_Avg    4
17     Radiacion_global_MJm2    4
18                    CVMeta   14
19              Invalid_Wind    4
20                  TT_C_Avg    4
21  Temperatura_superficie_C    4
22   Flujo_energia_suelo_Wm2    4
23        Radiacion_neta_Wm2    4
24               CNR_Wm2_Avg    4
25 Temperatura_termocupla1_C    4
26 Temperatura_termocupla2_C    4
27    Humedad_suelo_5cm_m3m3    4
28   Humedad_suelo_40cm_m3m3    4
\end{verbatim}

\textbf{Resumen de flags por variable}

\begin{verbatim}
# A tibble: 1 x 15
  temp_ok temp_fuera temp_na hr_ok hr_fuera hr_na pp_ok pp_negativa pp_na ws_ok
    <int>      <int>   <int> <int>    <int> <int> <int>       <int> <int> <int>
1    2434          0       4  2434        0     4  2434           0     4  2434
# i 5 more variables: ws_negativa <int>, ws_na <int>, rn_ok <int>,
#   rn_fuera <int>, rn_na <int>
\end{verbatim}

\begin{verbatim}
# A tibble: 1 x 15
  temp_ok temp_fuera temp_na hr_ok hr_fuera hr_na pp_ok pp_negativa pp_na ws_ok
    <dbl>      <dbl>   <dbl> <dbl>    <dbl> <dbl> <dbl>       <dbl> <dbl> <dbl>
1    99.8          0   0.164  99.8        0 0.164  99.8           0 0.164  99.8
# i 5 more variables: ws_negativa <dbl>, ws_na <dbl>, rn_ok <dbl>,
#   rn_fuera <dbl>, rn_na <dbl>
\end{verbatim}

\textbf{Resultados del control de calidad}

\begin{Shaded}
\begin{Highlighting}[]
\NormalTok{resultados\_cc }\OtherTok{\textless{}{-}} \FunctionTok{control\_calidad}\NormalTok{(datos\_brutos)}

\NormalTok{fuera\_rango\_df }\OtherTok{\textless{}{-}}\NormalTok{ purrr}\SpecialCharTok{::}\FunctionTok{map\_dfr}\NormalTok{(}
\NormalTok{resultados\_cc}\SpecialCharTok{$}\NormalTok{fuera\_rango,}
\SpecialCharTok{\textasciitilde{}} \FunctionTok{as.data.frame}\NormalTok{(.x),}
\AttributeTok{.id =} \StringTok{"variable"}
\NormalTok{)}

\NormalTok{atipicos\_iqr\_df }\OtherTok{\textless{}{-}}\NormalTok{ purrr}\SpecialCharTok{::}\FunctionTok{map\_dfr}\NormalTok{(}
\NormalTok{resultados\_cc}\SpecialCharTok{$}\NormalTok{atipicos\_iqr,}
\SpecialCharTok{\textasciitilde{}} \FunctionTok{as.data.frame}\NormalTok{(.x),}
\AttributeTok{.id =} \StringTok{"variable"}
\NormalTok{)}

\NormalTok{completitud\_df }\OtherTok{\textless{}{-}} \FunctionTok{data.frame}\NormalTok{(}
\AttributeTok{variable =} \FunctionTok{names}\NormalTok{(resultados\_cc}\SpecialCharTok{$}\NormalTok{completitud),}
\AttributeTok{completitud =} \FunctionTok{as.numeric}\NormalTok{(resultados\_cc}\SpecialCharTok{$}\NormalTok{completitud)}
\NormalTok{)}

\FunctionTok{list}\NormalTok{(}
\AttributeTok{fuera\_rango =}\NormalTok{ fuera\_rango\_df,}
\AttributeTok{atipicos\_iqr =}\NormalTok{ atipicos\_iqr\_df,}
\AttributeTok{completitud =}\NormalTok{ completitud\_df,}
\AttributeTok{saltos\_tiempo =}\NormalTok{ resultados\_cc}\SpecialCharTok{$}\NormalTok{saltos\_temporales}
\NormalTok{)}
\end{Highlighting}
\end{Shaded}

\begin{verbatim}
$fuera_rango
                 variable n_fuera_rango porcentaje
1      Temperatura_aire_C             0          0
2   Humedad_relativa_porc             0          0
3     Velocidad_viento_ms             0          0
4    Radiacion_global_Wm2             0          0
5        Precipitacion_mm             0          0
6  Humedad_suelo_5cm_m3m3             0          0
7 Humedad_suelo_40cm_m3m3             0          0

$atipicos_iqr
                    variable n_atipicos porcentaje
1       Radiacion_global_Wm2          0       0.00
2           Precipitacion_mm        179       7.34
3               Rayos_conteo          3       0.12
4        Velocidad_viento_ms         10       0.41
5    Direccion_viento_grados          0       0.00
6                  Rachas_ms          4       0.16
7         Temperatura_aire_C          0       0.00
8   Presion_atmosferica_mbar         28       1.15
9      Humedad_relativa_porc          0       0.00
10     Radiacion_global_MJm2          0       0.00
11  Temperatura_superficie_C          0       0.00
12   Flujo_energia_suelo_Wm2          0       0.00
13        Radiacion_neta_Wm2         66       2.71
14 Temperatura_termocupla1_C          0       0.00
15 Temperatura_termocupla2_C          0       0.00
16    Humedad_suelo_5cm_m3m3        190       7.79
17   Humedad_suelo_40cm_m3m3          0       0.00

$completitud
                    variable completitud
1       Radiacion_global_Wm2      100.00
2           Precipitacion_mm       99.84
3               Rayos_conteo       99.84
4        Velocidad_viento_ms       99.84
5    Direccion_viento_grados       99.84
6                  Rachas_ms       99.84
7         Temperatura_aire_C       99.84
8   Presion_atmosferica_mbar       99.84
9      Humedad_relativa_porc       99.84
10     Radiacion_global_MJm2       99.84
11  Temperatura_superficie_C       99.84
12   Flujo_energia_suelo_Wm2       99.84
13        Radiacion_neta_Wm2       99.84
14 Temperatura_termocupla1_C       99.84
15 Temperatura_termocupla2_C       99.84
16    Humedad_suelo_5cm_m3m3       99.84
17   Humedad_suelo_40cm_m3m3       99.84

$saltos_tiempo
[1] 103
\end{verbatim}

\section{Visualización gráfica de las
variables}\label{visualizaciuxf3n-gruxe1fica-de-las-variables}

\textbf{Series temporales facetadas de variables principales}

\pandocbounded{\includegraphics[keepaspectratio]{Calidad_datos_files/figure-pdf/unnamed-chunk-3-1.pdf}}

\textbf{Humedad del suelo en el tiempo (2 profundidades)}

\pandocbounded{\includegraphics[keepaspectratio]{Calidad_datos_files/figure-pdf/unnamed-chunk-4-1.pdf}}

\textbf{Precipitación diaria acumulada}

\begin{figure}[H]

{\centering \pandocbounded{\includegraphics[keepaspectratio]{Calidad_datos_files/figure-pdf/graf-precipitacion-diaria-1.pdf}}

}

\caption{Precipitación diaria acumulada.}

\end{figure}%

\textbf{Rosa de vientos}

\begin{figure}[H]

{\centering \pandocbounded{\includegraphics[keepaspectratio]{Calidad_datos_files/figure-pdf/graf-rosa-vientos-1.pdf}}

}

\caption{Rosa de vientos (dirección y velocidad del viento).}

\end{figure}%

\textbf{Distribución de temperaturas}

\begin{figure}[H]

{\centering \pandocbounded{\includegraphics[keepaspectratio]{Calidad_datos_files/figure-pdf/graf-distribucion-temperaturas-1.pdf}}

}

\caption{Distribución de temperaturas del aire, superficie y
termocuplas.}

\end{figure}%

\textbf{Generación de base de datos validada}

\begin{Shaded}
\begin{Highlighting}[]
\NormalTok{datos\_validados }\OtherTok{\textless{}{-}}\NormalTok{ datos\_qc }\SpecialCharTok{\%\textgreater{}\%}
\FunctionTok{mutate}\NormalTok{(}
\AttributeTok{Temperatura\_aire\_C =} \FunctionTok{if\_else}\NormalTok{(flag\_temp\_rango }\SpecialCharTok{==} \StringTok{"ok"}\NormalTok{, Temperatura\_aire\_C, }\ConstantTok{NA\_real\_}\NormalTok{),}
\AttributeTok{Humedad\_relativa\_porc =} \FunctionTok{if\_else}\NormalTok{(flag\_hr\_rango }\SpecialCharTok{==} \StringTok{"ok"}\NormalTok{, Humedad\_relativa\_porc, }\ConstantTok{NA\_real\_}\NormalTok{),}
\AttributeTok{Precipitacion\_mm =} \FunctionTok{if\_else}\NormalTok{(flag\_pp\_rango }\SpecialCharTok{==} \StringTok{"ok"}\NormalTok{, Precipitacion\_mm, }\ConstantTok{NA\_real\_}\NormalTok{),}
\AttributeTok{Velocidad\_viento\_ms =} \FunctionTok{if\_else}\NormalTok{(flag\_ws\_rango }\SpecialCharTok{==} \StringTok{"ok"}\NormalTok{, Velocidad\_viento\_ms, }\ConstantTok{NA\_real\_}\NormalTok{),}
\AttributeTok{Radiacion\_neta\_Wm2 =} \FunctionTok{if\_else}\NormalTok{(flag\_rn\_rango }\SpecialCharTok{==} \StringTok{"ok"}\NormalTok{, Radiacion\_neta\_Wm2, }\ConstantTok{NA\_real\_}\NormalTok{)}
\NormalTok{)}

\NormalTok{readr}\SpecialCharTok{::}\FunctionTok{write\_csv}\NormalTok{(datos\_validados, }\StringTok{"Datos\_Parcela\_3\_validado.csv"}\NormalTok{)}

\NormalTok{datos\_validados }\SpecialCharTok{\%\textgreater{}\%}
\FunctionTok{summarise}\NormalTok{(}
\AttributeTok{n\_temp\_validas =} \FunctionTok{sum}\NormalTok{(}\SpecialCharTok{!}\FunctionTok{is.na}\NormalTok{(Temperatura\_aire\_C)),}
\AttributeTok{n\_hr\_validas =} \FunctionTok{sum}\NormalTok{(}\SpecialCharTok{!}\FunctionTok{is.na}\NormalTok{(Humedad\_relativa\_porc)),}
\AttributeTok{n\_pp\_validas =} \FunctionTok{sum}\NormalTok{(}\SpecialCharTok{!}\FunctionTok{is.na}\NormalTok{(Precipitacion\_mm)),}
\AttributeTok{n\_ws\_validas =} \FunctionTok{sum}\NormalTok{(}\SpecialCharTok{!}\FunctionTok{is.na}\NormalTok{(Velocidad\_viento\_ms)),}
\AttributeTok{n\_rn\_validas =} \FunctionTok{sum}\NormalTok{(}\SpecialCharTok{!}\FunctionTok{is.na}\NormalTok{(Radiacion\_neta\_Wm2))}
\NormalTok{)}
\end{Highlighting}
\end{Shaded}

\begin{verbatim}
# A tibble: 1 x 5
  n_temp_validas n_hr_validas n_pp_validas n_ws_validas n_rn_validas
           <int>        <int>        <int>        <int>        <int>
1           2434         2434         2434         2434         1821
\end{verbatim}

\begin{center}\rule{0.5\linewidth}{0.5pt}\end{center}

\texttt{\{\#\}}




\end{document}
